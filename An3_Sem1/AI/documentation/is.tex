\documentclass[a4paper,12pt]{report}
\usepackage{algorithmic}
\usepackage[linesnumbered,ruled,vlined]{algorithm2e}
\usepackage[margin=2cm]{geometry}
\usepackage[utf8]{inputenc}
\usepackage{listings} 
\usepackage{graphicx} 
\usepackage{color}
\usepackage{xcolor}
\usepackage{hyperref}
\usepackage{verbatim}
%\usepackage{mdframed}

\definecolor{codegreen}{rgb}{0,0.6,0}
\definecolor{codegray}{rgb}{0.5,0.5,0.5}
\definecolor{codepurple}{rgb}{0.58,0,0.82}
\definecolor{backcolour}{rgb}{0.95,0.95,0.92}

\lstdefinestyle{codeBoxStyle}{
    backgroundcolor=\color{backcolour},   
    commentstyle=\color{codegreen},
    keywordstyle=\color{magenta},
    numberstyle=\tiny\color{codegray},
    stringstyle=\color{codepurple},
    basicstyle=\ttfamily\footnotesize,
    breakatwhitespace=false,         
    breaklines=true,                 
    captionpos=b,                    
    keepspaces=true,                 
    numbers=left,                    
    numbersep=5pt,                  
    showspaces=false,                
    showstringspaces=false,
    showtabs=false,                  
    tabsize=2
}

\newcommand{\currentdata}{14 February 2015}
\newtheorem{example}{Example}

\begin{document}
\vspace{-5cm}
\begin{center}
Department of Computer Science\\
Technical University of Cluj-Napoca\\
\includegraphics[width=10cm]{fig/footer}
\end{center}
\vspace{1cm}
%\maketitle
\begin{center}
\begin{Large}
 \textbf{Artificial Intelligence}\\
\end{Large}
\textit{Laboratory activity}\\
\vspace{3cm}
Name:\\
Podina Tudor\\
Tudor Roxana\\
\vspace{1cm}
Group:\\
30434\\
\vspace{1cm}
Email:\\
tpodina@gmail.com / Podina.Do.Tudor@student.utcluj.ro\\
roxi13.tudor@gmail.com / Tudor.Co.Roxana@student.utcluj.ro\\
\vspace{8cm}
Teaching Assistant: Adrian Groza\\
Adrian.Groza@cs.utcluj.ro\\
\vspace{1cm}
\includegraphics[width=10cm]{fig/footer}
\end{center}

\tableofcontents

\input{policy}

%\chapter{Laboratory works}

\chapter{A1: Search - Pacman Wants a Family}
\section{Preview}
\textbf{What was the goal?} The goal of this assignment was to allow us to take a deep dive into the Pacman framework.\\
In doing so, we had the opportunity to understand how each method gets called and when. Moreover, we got to experience first-hand
what happens under the hood of the framework.\\
Besides the framework-side of the assignment, we were also given the chance to experiment with multiagents and how they
work, as well as research how NPC's work, not just within the Pacman game, but as a whole.\\
As a final mention, we also wanted to do something extra and had the chance to read some PhD papers. And while they
proved too difficult to implement on short notice (some were even outside the subject of this semster), they proved to be
very insightful.
\section{Keyboard agent}
\subsection{Gameplay}
We decided on switching the role of Pacman from being chased to chasing either Ms. Pacman or food.\\
This allowed to introduce Ms. Pacman as an agent that could be controlled via keyboard. We also focused
on giving Ms. Pacman a hard time by allowing Pacman to eat her food, thus lowering her score.
\subsection{Objective}
\textbf{Survive.} On a serious note, what the player (Ms. Pacman) has to do is very similar to the default
version of Pacman: eat food and avoid ghosts. There is an improvement with respect to the challenge, since
Ms. Pacman also has to deal with Pacman eating her food, causing her to lose out on points.
\section{Multiagents}
\subsection{Objective}
As mentioned before, Pacman gets an upgrade within our version of the game, meaning that he is the "antagonist"
now. We decided to replace the concept of ghosts with Pacmen that share a common goal of eating as much as possible
and chasing Ms. Pacman.
\subsection{How do they work?}
At first, we needed to figure out how to compute the next steps based on the current location of Ms. Pacman.
With this in mind, we explored the concept of \emph{ghost agents} which have a \emph{getDistribution} method.
This method generates a distribution based on a list of the best actions that Pacman can take*.\\
We also focused on having two types of agents. One of them works using the \emph{manhattan distance} while the other uses the \emph{maze distance} towards a goal.
It is also worth noting that we tweaked the goal to be composed of food and Ms. Pacman. This means that Pacman either chases
food or Ms. Pacman depending on who is closer. Also, we played a little with the idea of attaching a probability of being chosen
as a goal for each of the two targets. While our implementation allows this, we did not have enough time to implement
and test this properly, so we will leave it for a future improvement.\\
Note*: There is something in the framework that we decided not to change, but we are aware of it: the action that
Pacman eventually takes is randomly picked.
\chapter{A2. Planning}
\section{How we chose the task}
Initially, we wanted to combine the idea of having Ms. Pacman with random maze generation and junior Pacmen.
Instead of overloading our assignment with ambitious ideas that we may or may not have implemented properly,
we decided to stick with adding Ms. Pacman as the agent controlled by the player and multiple Pacmen replacing
the default ghosts.\\
After some brainstorming, we also decided on having Pacman eat Ms. Pacman's food.
\section{Sub-task distribution}
At first, we did some pair programming, mostly to setup the project, do the necessary research and explore the
framework to decide upon what needs to be done and what is already implemented.\\
After that, we tried to parallelize the development process and distributed the tasks as such: \begin{enumerate}
    \item Random maze generation - discarded - Podina. T
    \item Ms. Pacman and Pacmen animations - Tudor. R
    \item Multiagents decision making - Podina. T
    \item Multiagents food eating - Tudor. R
    \item User experience improvements - Podina. T, Tudor R.
    \item Documentation: \begin{enumerate}
        \item Writing and structuring - Podina. T
        \item Quality control and refactoring - Tudor. R
    \end{enumerate}
\end{enumerate}

\bibliographystyle{plain}
\bibliography{is}
\begin{enumerate}
    \item \url{http://ai.berkeley.edu/multiagent.html} - base project
    \item \url{https://github.com/lb5160482/Pacman-Search} \label{search_github_repo} - useful for examples and PositionSearchProblem which we used for the Pacmen agents
\end{enumerate}

\appendix

\chapter{Your original code}
Don't be a cheater! Cheating affects your colleagues, scholarships and a lot more.
This section should contain only code developed by you, without any line re-used from other sources. 
This section helps me to correctly evaluate your amount of work and results obtained. 

\vspace{0.5cm}
\lstset{style=codeBoxStyle}
\lstinputlisting[language=python]{code/MazeDirectionalPacman.py}
\vspace{0.5cm}
Note 1: DirectionalPacman works the same, only it uses the manhattan istance, rather than the maze distance.\\
Note 2: The mazeDistance() method is defined in the repository mentioned here \ref{search_github_repo}. We used
helpers from the said repository and made a slight adjustment to the method to work with our code. Here is the
implementation from our code.
\vspace{0.5cm}
\lstset{style=codeBoxStyle}
\lstinputlisting[language=python]{code/MazeDistance.py}
\vspace{0.5cm}
Note 3: as a future improvement to this method, it would be interesting to allow the user to set which search
algorithm to use for the mazeDistance() method. By default, we are using the Iterative Deepening Search. Maybe
the search algorithm would be swapped for a better one as the user sets a higher difficutly to the game.

\vspace{2cm}
\begin{center}
Intelligent Systems Group\\
\includegraphics[width=10cm]{fig/footer}
\end{center}



\end{document}
